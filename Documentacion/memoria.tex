\documentclass[12pt,a4paper]{article}
% ------------------------------------------------
%                 PACKAGES
% ------------------------------------------------
\usepackage[utf8]{inputenc}
\usepackage[T1]{fontenc}
\usepackage[spanish]{babel}
\usepackage[a4paper,margin=2.5cm]{geometry}
\usepackage{graphicx}
\usepackage{hyperref}
\usepackage{fancyhdr}
\usepackage{titlesec}
\usepackage{color}
\usepackage{xcolor}
\usepackage{array}
\usepackage{longtable}
\usepackage{booktabs}
\usepackage{enumitem}
\usepackage{multicol}
\usepackage{parskip}

% ------------------------------------------------
%              GLOBAL  SETTINGS
% ------------------------------------------------
\hypersetup{
    pdftitle={Memoria del Proyecto - StatsFootPRO},
    pdfauthor={Equipo de Desarrollo StatsFootPRO},
    pdfsubject={Memoria técnica y funcional del proyecto StatsFootPRO},
    pdfkeywords={Flutter, Supabase, OneSignal, Clean Architecture, Riverpod, AutoRoute},
    colorlinks=true,
    linkcolor=blue,
    citecolor=blue,
    urlcolor=blue
}

% Colores corporativos
\definecolor{primary}{HTML}{0057FF}
\definecolor{secondary}{HTML}{00C1D4}
\definecolor{accent}{HTML}{FF6B00}

% Encabezado y pie de página
\pagestyle{fancy}
\fancyhf{}
\fancyhead[L]{\textbf{StatsFootPRO}}
\fancyhead[R]{Memoria del Proyecto}
\fancyfoot[C]{\thepage}
\renewcommand{\headrulewidth}{0.4pt}
\renewcommand{\footrulewidth}{0.4pt}

% Formato de títulos
\titleformat{\section}{\Large\bfseries\color{primary}}{\thesection.}{0.5em}{}
\titleformat{\subsection}{\large\bfseries\color{secondary}}{\thesubsection.}{0.5em}{}

% ------------------------------------------------
%                  DOCUMENT
% ------------------------------------------------
\begin{document}

% ------------------------------------------------
%                   COVER
% ------------------------------------------------
\begin{titlepage}
  \centering
  \vspace*{4cm}
  {\Huge \textbf{Memoria del Proyecto}\\[0.5cm]}
  {\LARGE StatsFootPRO}\\[1.5cm]
  \includegraphics[width=0.6\linewidth]{logo_placeholder.png}\\[1.5cm]
  {\large \textbf{Versión}: 1.0.0}\\[0.5cm]
  {\large Fecha: \today}\\[4cm]
  \begin{flushright}
    {\large Equipo de Desarrollo StatsFootPRO}\\
  \end{flushright}
  \vfill
\end{titlepage}

% Índice
\tableofcontents
\newpage

% ------------------------------------------------
%            1. Introducción
% ------------------------------------------------
\section{Introducción}
StatsFootPRO es una aplicación móvil desarrollada en Flutter que facilita la gestión de partidos de fútbol amateur, permitiendo a los usuarios crear, unirse y administrar encuentros, así como invitar amigos y recibir notificaciones en tiempo real. Esta memoria documenta el proceso de desarrollo, las decisiones arquitectónicas, la implementación técnica y las pruebas realizadas.

% ------------------------------------------------
%            2. Objetivos del Proyecto
% ------------------------------------------------
% ------------------------------------------------
%            2. Justificación
% ------------------------------------------------
\section{Justificación}
El desarrollo de \textbf{StatsFootPRO} responde a la necesidad detectada en la comunidad del fútbol amateur de disponer de una herramienta integral que simplifique la coordinación de partidos, permita el seguimiento estadístico y mejore la experiencia social de los jugadores. Actualmente, las opciones existentes son genéricas o están fragmentadas en varias aplicaciones, lo que genera fricción y desorganización. Esta plataforma, centrada en el usuario hispanohablante, supone:
\begin{itemize}[leftmargin=*]
  \item Reducir el tiempo y el esfuerzo necesarios para organizar encuentros deportivos.
  \item Proporcionar un registro histórico de rendimiento accesible y motivador para los jugadores.
  \item Fomentar la práctica deportiva amateur y la creación de comunidades locales.
  \item Ofrecer una base escalable para introducir funcionalidades premium y monetización futura.
\end{itemize}

% ------------------------------------------------
%            3. Alcance
% ------------------------------------------------
\section{Alcance}
\subsection{Incluido en el Proyecto}
\begin{itemize}[leftmargin=*]
  \item Sistema de autenticación y recuperación de contraseñas con Supabase.
  \item Gestión completa de partidos: creación, edición, invitaciones y confirmaciones.
  \item Notificaciones push en tiempo real mediante OneSignal.
  \item Estadísticas básicas de partidos y jugadores.
  \item Interfaz bilingüe (español e inglés) optimizada para iOS y Android.
  \item Arquitectura limpia con escalabilidad garantizada.
\end{itemize}
\subsection{Excluido del Proyecto}
\begin{itemize}[leftmargin=*]
  \item Organización de ligas o torneos avanzados.
  \item Funcionalidades premium de pago y marketplace.
  \item Integraciones con plataformas de streaming.
\end{itemize}

% ------------------------------------------------
%            4. Metodología
% ------------------------------------------------
\section{Metodología}
Se emplea un enfoque \textbf{ágil} basado en \textbf{Scrum} con sprints de dos semanas. Las prácticas adoptadas incluyen:
\begin{itemize}[leftmargin=*]
  \item \textbf{Planificación de sprint}: definición de objetivos y tareas priorizadas.
  \item \textbf{Reuniones diarias}: seguimiento del avance y bloqueo rápido de impedimentos.
  \item \textbf{Revisión de sprint}: demostración de funcionalidades y recogida de feedback.
  \item \textbf{Retrospectiva}: mejora continua de procesos y herramientas.
  \item \textbf{CI/CD}: integración continua con pruebas automáticas y despliegue en Netlify/Firebase.
\end{itemize}

% ------------------------------------------------
%            5. Recursos
% ------------------------------------------------
\section{Recursos}
\subsection{Recursos Humanos}
\begin{longtable}{@{}p{5cm}p{8cm}@{}}
\toprule
\textbf{Rol} & \textbf{Responsabilidades Principales} \\
\midrule
Desarrollador Flutter Senior & Arquitectura, desarrollo de funciones core, revisión de código. \\
Desarrollador Flutter Junior & Implementación de UI, pruebas y soporte. \\
Diseñador UI/UX & Prototipado en Figma, pruebas de usabilidad, guía de estilos. \\
Product Owner & Gestión de requisitos, backlog y roadmap. \\
\bottomrule
\end{longtable}

\subsection{Recursos Técnicos}
\begin{itemize}[leftmargin=*]
  \item MacBook Pro y PCs Windows para desarrollo.
  \item Dispositivos iOS y Android para pruebas físicas.
  \item Herramientas: VS Code, Android Studio, GitHub, Figma.
  \item Servicios: Supabase (plan gratuito inicial), OneSignal, Netlify, Firebase Crashlytics.
\end{itemize}

\subsection{Recursos Económicos}
\begin{longtable}{@{}p{6cm}p{3cm}p{3cm}@{}}
\toprule
\textbf{Categoría} & \textbf{Monto (EUR)} & \textbf{Porcentaje} \\
\midrule
Desarrollo & 22\,000 & 68.75\% \\
Diseño & 4\,000 & 12.50\% \\
Marketing & 3\,000 & 9.38\% \\
Infraestructura & 1\,500 & 4.69\% \\
Contingencia & 1\,500 & 4.69\% \\
\textbf{Total} & \textbf{32\,000} & \textbf{100\%} \\
\bottomrule
\end{longtable}

\begin{figure}[h]
  \centering
  \includegraphics[width=0.6\linewidth]{budget_pie_placeholder.png}
  \caption{Distribución presupuestaria estimada}
\end{figure}

% ------------------------------------------------
%            6. Cronograma
% ------------------------------------------------
\section{Cronograma}
El proyecto se desarrolló en cinco fases a lo largo de \textbf{tres meses}.
\begin{enumerate}[leftmargin=*]
  \item \textbf{Planificación y Diseño} (Semanas 1--2): definición de requisitos y diseño UI/UX.
  \item \textbf{Desarrollo Core} (Semanas 3--6): autenticación, perfil y partidos.
  \item \textbf{Funcionalidades Avanzadas} (Semanas 7--9): notificaciones push y estadísticas.
  \item \textbf{Pruebas y Optimización} (Semanas 10--11): pruebas de rendimiento, corrección de errores.
  \item \textbf{Lanzamiento y Seguimiento} (Semana 12): publicación en tiendas y monitoreo inicial.
\end{enumerate}

\begin{figure}[h]
  \centering
  \includegraphics[width=0.9\linewidth]{timeline_placeholder.png}
  \caption{Cronograma general del proyecto}
\end{figure}

% ------------------------------------------------
%            7. Resultados Esperados
% ------------------------------------------------
\section{Resultados Esperados}
\begin{itemize}[leftmargin=*]
  \item Aplicación estable publicada en iOS y Android con \textgreater{}5\,000 usuarios en 3 meses.
  \item Valoración media \textgreater{}4.5/5 estrellas en tiendas.
  \item Tasa de retención del 40\% a 30 días.
  \item Tiempo de respuesta del servidor \textless{}200~ms en el 95\% de las solicitudes.
  \item Base sólida para introducir funcionalidades premium en el futuro.
\end{itemize}

% ------------------------------------------------
%            8. Conclusiones y Recomendaciones
% ------------------------------------------------
\section{Conclusiones y Recomendaciones}
\textbf{StatsFootPRO} ofrece una solución diferenciada en el mercado del fútbol amateur gracias a su arquitectura escalable, experiencia de usuario cuidada y enfoque en la comunidad. Se recomienda:
\begin{itemize}[leftmargin=*]
  \item Explorar integraciones con reserva de canchas para ampliar el ecosistema.
  \item Analizar modelos de monetización freemium basados en funcionalidades premium.
  \item Mantener ciclos de retroalimentación continua con usuarios para priorizar mejoras.
\end{itemize}

% ------------------------------------------------
%            9. Objetivos del Proyecto
% ------------------------------------------------
\section{Objetivos del Proyecto}
\begin{itemize}[leftmargin=*]
  \item Disponibilizar una plataforma móvil intuitiva para la organización de partidos. 
  \item Garantizar un flujo de autenticación y recuperación de contraseñas seguro mediante Supabase. 
  \item Implementar notificaciones push reales usando OneSignal. 
  \item Mantener una arquitectura limpia y escalable basada en principios SOLID.
  \item Ofrecer una experiencia de usuario moderna con animaciones y UI responsiva.
\end{itemize}

% ------------------------------------------------
%            3. Arquitectura
% ------------------------------------------------
\section{Arquitectura}
La aplicación sigue el patrón \textbf{Clean Architecture}, separando claramente las capas de presentación, dominio y datos. El manejo de estados se gestiona con \textbf{Riverpod}, las rutas con \textbf{AutoRoute} y la inyección de dependencias con \textbf{getIt}. Para la persistencia y autenticación se utiliza \textbf{Supabase}, mientras que \textbf{OneSignal} gestiona las notificaciones push.

\begin{figure}[h]
  \centering
  \includegraphics[width=0.9\linewidth]{architecture_placeholder.png}
  \caption{Visión general de la arquitectura}
\end{figure}

% ------------------------------------------------
%            4. Tecnologías y Herramientas
% ------------------------------------------------
\section{Tecnologías y Herramientas}
\begin{longtable}{@{}p{4cm}p{10cm}@{}}
  \toprule
  \textbf{Tecnología} & \textbf{Uso principal} \\
  \midrule
  Flutter & Desarrollo de la interfaz de usuario multiplataforma \\
  Dart & Lenguaje de programación principal \\
  Supabase & Backend como servicio para autenticación y base de datos Postgres \\
  Riverpod & Gestión de estado reactiva y escalable \\
  AutoRoute & Enrutamiento y deep linking \newline type–safe \\
  OneSignal & Envío de notificaciones push \\
  getIt & Inyección de dependencias \\
  LaTeX & Generación de documentación profesionales \\
  GitHub Actions & Integración continua y despliegue automático \\
  Netlify & Hosting del front-end web y redirecciones seguras \\
  \bottomrule
\end{longtable}

% ------------------------------------------------
%            5. Módulos Clave Implementados
% ------------------------------------------------
\section{Módulos Clave Implementados}
\subsection{Flujo de Recuperación de Contraseña}
Se implementó una solución robusta basada en Supabase que incluye:
\begin{enumerate}[leftmargin=*]
  \item Pantalla de solicitud de restablecimiento con validación de correo y feedback amigable.
  \item Redirección segura vía Netlify evitando el consumo prematuro de tokens.
  \item Pantalla de actualización de contraseña con validación de sesión mediante \texttt{exchangeCodeForSession}.
  \item Manejo de errores exhaustivo y mensajes bilingües.
\end{enumerate}

\subsection{Notificaciones Push Reales}
\begin{itemize}[leftmargin=*]
  \item Integración con la API REST de OneSignal mediante el paquete \texttt{http}.
  \item Soporte para envío a usuarios específicos usando \texttt{playerIds} o \texttt{externalUserId}.
  \item Pantalla de prueba mejorada para monitorear el resultado del envío.
\end{itemize}

\subsection{Formateo de Fechas}
Se añadió un método centralizado que formatea las fechas como \texttt{dd/MM/yyyy HH:mm} utilizando el paquete \texttt{intl}, garantizando coherencia y legibilidad para el público hispanohablante.

% ------------------------------------------------
%            6. Seguridad
% ------------------------------------------------
\section{Seguridad}
\begin{itemize}[leftmargin=*]
  \item Almacenamiento seguro de claves API en variables de entorno y constantes dentro del servicio.
  \item Uso de sesiones protegidas y expiración automática tras operaciones sensibles.
  \item Validaciones de entrada extensivas para prevenir ataques de inyección y asegurar datos.
\end{itemize}

% ------------------------------------------------
%            7. Pruebas
% ------------------------------------------------
\section{Pruebas}
Se llevaron a cabo pruebas unitarias y de aceptación siguiendo las convenciones \textbf{Arrange–Act–Assert} y \textbf{Given–When–Then}. Además, se implementaron mock objects para aislar dependencias costosas. Los flujos críticos como la recuperación de contraseña y las notificaciones push cuentan con pruebas automatizadas en GitHub Actions.

% ------------------------------------------------
%            8. Despliegue y CI/CD
% ------------------------------------------------
\section{Despliegue y CI/CD}
El proyecto se integra con \textbf{GitHub Actions} para compilar y ejecutar pruebas en cada \texttt{push}. La aplicación web se despliega automáticamente en \textbf{Netlify}, mientras que las versiones móviles se distribuyen a través de \textbf{Firebase App Distribution}. Las variables de entorno se gestionan con seguridad mediante secretos de repositorio.

% ------------------------------------------------
%            9. Futuro Trabajo
% ------------------------------------------------
\section{Futuro Trabajo}
\begin{itemize}[leftmargin=*]
  \item Implementar chat en tiempo real entre jugadores usando Supabase Realtime.
  \item Añadir analíticas avanzadas con Firebase Analytics.
  \item Desarrollar un panel web administrativo para gestionar ligas y estadísticas.
  \item Internacionalizar la aplicación a más idiomas (inglés, portugués, francés).
\end{itemize}

% ------------------------------------------------
%            10. Conclusión
% ------------------------------------------------
\section{Conclusión}
StatsFootPRO proporciona una solución completa y moderna para la gestión de partidos de fútbol amateur. Gracias a una arquitectura limpia, integración con servicios de backend robustos y una experiencia de usuario cuidada, se sienta una base sólida para futuras funcionalidades y escalabilidad.

\vspace{1cm}
\begin{center}
  \textit{“El fútbol es la excusa perfecta para unir a las personas; StatsFootPRO es la herramienta para lograrlo.”}
\end{center}

\end{document}
